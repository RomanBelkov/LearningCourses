% Тут используется класс, установленный на сервере Papeeria. На случай, если
% текст понадобится редактировать где-то в другом месте, рядом лежит файл matmex-diploma-custom.cls
% который в момент своего создания был идентичен классу, установленному на сервере.
% Для того, чтобы им воспользоваться, замените matmex-diploma на matmex-diploma-custom
% Если вы работаете исключительно в Papeeria то мы настоятельно рекомендуем пользоваться
% классом matmex-diploma, поскольку он будет автоматически обновляться по мере внесения корректив
%

% По умолчанию используется шрифт 14 размера. Если нужен 12-й шрифт, уберите опцию [14pt]
% \documentclass[14pt]{matmex-diploma}
\documentclass[14pt]{matmex-diploma-custom}

\begin{document}
% Год, город, название университета и факультета предопределены,
% но можно и поменять.
% Если англоязычная титульная страница не нужна, то ее можно просто удалить.
\filltitle{ru}{
    chair              = {Программная инженерия},
    title              = {Интеграция программирования на языке Python в образовательные решения TRIK},
    % Здесь указывается тип работы. Возможные значения:
    %   coursework - Курсовая работа
    %   diploma - Диплом специалиста
    %   master - Диплом магистра
    %   bachelor - Диплом бакалавра
    type               = {bachelor},
    position           = {студента},
    group              = 471,
    author             = {Белков Роман Владимирович},
    supervisorPosition = {ст.\, преп.\,},
    supervisor         = {Кириленко Я.\,А.},
    reviewerPosition   = {OOO "ИнтеллиДжей Лабс"\\ программист},
    reviewer           = {Мордвинов Д.\,А.},
    chairHeadPosition  = {д.\,ф.-м.\,н., профессор},
    chairHead          = {Терехов А.\,Н.},
%   university         = {Санкт-Петербургский Государственный Университет},
%   faculty            = {Математико-механический факультет},
%   city               = {Санкт-Петербург},
%   year               = {2013}
}
\filltitle{en}{
    chair              = {Software engineering},
    title              = {Integration of Python programming with TRIK educational solutions},
    author             = {Roman Belkov},
    supervisorPosition = {senior lecturer},
    supervisor         = {Iakov Kirilenko},
    reviewerPosition   = {IntelliJ Labs Co. Ltd\\ Software Engineer},
    reviewer           = {Dmitry Mordvinov},
    chairHeadPosition  = {professor},
    chairHead          = {Andrey Terekhov},
}
\maketitle
\tableofcontents
% У введения нет номера главы
\section*{Введение}
По своим вычислительным ресурсам робототехнические контроллеры, доступные широкому кругу пользователей, приближаются к показателям персональных компьютеров пятилетней давности. Такая тенденция позволяет постепенно применять в разработке современные методологии и технологии наравне с классическими для контроллеров низкоуровневыми языками и технологиями. Поскольку робототехника активно используется для STEM \cite{stemEducation,stemRobotics} образования, внедрение популярных технологий, использующихся в промышленном программировании, позволит методистам разрабатывать программы обучения, направленные на более широкий круг пользователей. Одной из популярной технологий, набравшей гигантскую популярность в образовательной сфере, является язык Python.

% \subsection*{Python}
Python в данный момент является вторым по популярности языком после Java согласно списку PyPL\footnote{pypl.github.io/PYPL.html}. В последние несколько лет Python стал активно внедряться в образовательные программы. Например, в Массачусетском технологическом институте, одном из ведущих университетов в области инженерного и технического образования, студентам первого года обучения читают курс "Introduction to Electrical Engineering and Computer Science I" \cite{stemMITCourse}, который представляет собой программирование роботов на языке Python. Другие зарубежные университеты тоже достаточно быстро перешли на использование Python в вводных курсах по программированию \cite{pythonUni}, тем самым обеспечив Python первенство среди языков программирования, использующихся в университетах США.

За трендом, установленным университетами, практически сразу же последовали школы \cite{stemSecCourse, stemSchool}, переходя на Python и внедряя новые курсы обучения программированию на языке Python.

% \subsection*{TRIK}
TRIK\footnote{www.trikset.com} -- это кибернетический контроллер, созданный с целью обучения программированию студентов и школьников. Данный контроллер примечателен тем, что обладает достаточными вычислительными мощностями для решения сложных робототехнических задач и реализации ресуркоёмких алгоритмов, а также отсутствием необходимости навыков пайки и электротехники. Для контроллера ТРИК существует среда TRIK Studio \cite{qrealRobots, TRIKStudioTech}, позволяющая облегчить знакомство с робототехникой школьникам младших и средних классов с использованием визуального программирования. Одними из наиболее значимых достоинств среды являются генераторы кода на текстовых языках программирования и интерпретатор текстового кода для 2D модели робота. Это позволяет преподавателям произвести более плавный переход от визуального программирования к текстовому и впоследствии обучать сложным синтаксическим конструкциям языков программирования. ПО контроллера TRIK и TRIK Studio образуют программное обеспечение образовательных решений ТРИК, которое на данный момент поддерживает следующие языки программирования: языки платформ Java и .NET, JavaScript, C++, Pascal. Добавив к перечисленным языкам Python, TRIK может считаться идеальной робототехнической платформой для обучения школьников и студентов.


\section*{Постановка задачи}

Целью данной квалификационной работы является разработка и внедрение системы, позволяющей использовать язык Python для обучения программированию роботов на базе контроллера ТРИК. Для достижения поставленной цели необходимо выполнить следующие задачи.

\begin{enumerate}
\item Изучить архитектуру существующего ПО образовательных решений ТРИК.
\item Определить требования к системе.
\item Разработать архитектуру системы.
\item Реализовать систему и внедрить его в существующие решения ТРИК.
\item Создать прототип пользовательской документации.
\end{enumerate}

\section{Обзор технологий}

Как правило, образовательное робототехническое решение состоит из следующих компонентов:
\begin{itemize}
    \item Методические материалы.
    \item Техническое решение.
        \begin{itemize}
            \item Аппаратная платформа.
            \item Конструктор.
            \item Программное обеспечение.
        \end{itemize}
\end{itemize}

\subsection{ТРИК}
ТРИК — это кибернетический контроллер под управлением операционной системы на основе ядра Linux. Он предназначен для управления роботами, беспилотными летательными аппаратами, средствами передвижения и встраиваемыми устройствами. 

В образовательных решениях ТРИК программное обеспечение представлено следующими компонентами:
\begin{itemize}
    \item Библиотека времени исполнения trikRuntime на роботе
    \item Среда программирования роботов TRIK Studio
    \item Интерпретатор 2D модели для проверки задач на Stepik
\end{itemize}

\subsubsection{trikRuntime}
trikRuntime -- это библиотека времени исполнения, позволяющая реализовывать алгоритмы, использующие контроллер и подключенную к нему периферию. Полная архитектура trikRuntime представлена на TODO рисунке

Библиотека trikControl предоставляет интерфейс для программирования роботов с помощью языков JavaScript или C++ с использованием Qt. Ниже перечислены некоторые классы trikControl: Brick — класс, отвечающий за контроллер, инициализирующий периферию робота и дающий к ней доступ, Sensor, ServoMotor и PowerMotor — классы, отвечающие за работу с сенсорами, сервомоторами и силовыми моторами, соответственно. Каждый из них даёт возможность прочитать или выставить значение, что выполняется путём записи или чтения значения в файле, отвечающем за соответствующее устройство.

% Идеал -- QtScript\footnote{doc.qt.io/qt-4.8/qtscript-module.html}, переименованный в QJSEngine\footnote{doc.qt.io/qt-5/qjsengine.html} с версии Qt 5.0 и уже используемый в основном фреймворке для контроллера ТРИК.

Подробно рассмотрим, как исполняется код на языке JavaScript с помощью компонентов trikRun и trikScriptRunner. trikScriptRunner использует технологию QtScript\footnote{doc.qt.io/qt-4.8/qtscript-module.html} (с версии Qt 5.0 -- QJSEngine\footnote{doc.qt.io/qt-5/qjsengine.html})

\subsubsection{TRIK Studio}
TRIK Studio -- это среда визуального и текстового программирования, поддерживающая наиболее популярные в России и Европе образовательные робототехнические платформы: Lego NXT, Lego EV3, TRIK, некоторые версии STM32. В активной стадии разработки -- поддержка большего количества платформ. Поддержка аппаратной платформы осуществляется специальными плагинами.

Помимо плагинов, поддерживающих генерацию или интерпретацию языков программирования, TRIK Studio обладает инструментами автоматизации и отладки: например, пользователь может из среды загрузить и запустить программу на роботе, а после запуска программы на роботе получать вывод программы в консоль TRIK Studio.

\subsubsection{TRIK Stepik solver}

\subsection{Фреймворки взаимодействия Python и C++}


Существует большое количество фреймворков, с использованием которых можно решить задачу использования существующего кода на C++ из языка Python. Поскольку ПО контроллера ТРИК и среда TRIK Studio являются свобободно распространяемым программным обеспечением, то в данной работе рассматривались только следующие фреймворки: PyQt, PySide, PythonQt, Boost.Python, SWIG.

Прежде, чем приступить к рассмотрению фреймворков взаимодействия, были выделены следующие требования, которым должен удовлетворять выбранный фреймворк.

\begin{itemize}
    \item Возможность поддержать исполнение Python из C++ приложения
    \item Поддержка Qt
    \item Поддержка фреймворка компанией или сообществом
    \item Доступная документация
\end{itemize}

\subsection{PyQt}

PyQt \cite{pyqt} -- это набор обёрток для фреймворка Qt, который позволяет использовать большую часть функциональности Qt в Python, тем самым достигая 

PyQt очень популярен и хорош для переиспользования существующих библиотек, написанных на C++ с использованием Qt из Python. PyQt не предоставляет возможностей для встраивания частей на языке Python в Qt-приложение. Для удобного переиспользования уже существующего кода на C++ существует инструмент SIP\footnote{www.riverbankcomputing.com/software/sip/intro}, который по специальным sip-файлам генерирует код обёрток на языке C++. Однако sip-файлы весьма неудобны тем, что они являются почти точными копиями заголовочных файлов. Это приводит к тому, что разработчикам приходиться изменять sip-файлы при каждом изменении C++ кода библиотеки. Более того, SIP обладает весьма слабым парсером языка C++, что не позволяет поддерживать некоторые конструкции динамически. Таким образом, приходиться прибегать к инъекциям дописанного руками кода в уже сгенерированный код, что является достаточно плохой практикой в силу затратности такого подхода и высокой вероятности допустить ошибку.

\subsection{PySide}
PySide -- это набор обёрток для фреймворка Qt. Проект PySide стартовал как аналог PyQt и практически не отличается от PyQt. Единственное значимое отличие -- PyQt распространяется под коммерческой или GPLv3 лицензией, в то время как PySide распространяется под LGPL лицензией, что позволяет использовать его с большей свободой. 

Также дополнительным плюсом является то, что Qt Company\footnote{en.wikipedia.org/wiki/The\_Qt\_Company} официально поддерживает PySide, что обеспечивает активную разработку фреймворка. Несмотря на это, документация у проекта всё равно находится в весьма начальной стадии.

\subsection{PythonQt}
Плюсы:
Возможна интерпретация Python из C++ приложения
Поддержка Qt
Проблемы:
Часть обёрток придётся писать руками

\subsection{Boost.Python}
Boost.Python \cite{abrahams2003boost}

Плюсы:
Возможна интерпретация Python из C++ приложения
Поддержка Qt
Проблемы:
Часть обёрток придётся писать руками

\subsection{SWIG}

\subsection{Выбор фреймворка}

Таблица \ref{table:frameworkComparison} на странице \pageref{table:frameworkComparison}

Сначала PyQt, потом PythonQt

% \begin{table}[]
% \centering
% \resizebox{\textwidth}{!}{%
% \begin{tabular}{|l|c|c|c|c|c|}
% \hline
%                                                                               & PyQt & PySide & PythonQt & SWIG & Boost.Python \\ \hline
% поддержка Qt                                                                  & +    & +      & +        & +/-  & -            \\ \hline
% \begin{tabular}[c]{@{}l@{}}поддержка фреймворка\\ разработчиками\end{tabular} & +    & -      & -        & +    & +            \\ \hline
% доступная документация                                                        & +    & -      & -        & +    & +            \\ \hline
% \end{tabular}%
% }
% \caption{Сравнение фреймворков}
% \label{table:frameworkComparison}
% \end{table}

\begin{table}[]
\centering
\resizebox{\textwidth}{!}{%
\begin{tabular}{|l|c|c|c|c|c|}
\hline
                                                                          & PyQt                                                          & PySide & PythonQt & SWIG & Boost.Python  \\ \hline
встраивание Python в C++                                                  & -                                                             & -      & +        & -    & +             \\ \hline
поддержка Qt                                                              & +                                                             & +      & +        & -    & -             \\ \hline
\begin{tabular}[c]{@{}l@{}}поддержка фреймворка\end{tabular} & +                                                             & -      & -        & +    & +             \\ \hline
хорошая документация                                                      & +                                                             & -      & -        & +    & +             \\ \hline
обёртка С++ для Python                                                    & +                                                             & +      & -        & +    & +             \\ \hline
лицензия                                                                  & \begin{tabular}[c]{@{}c@{}}GPL + \\ коммерческая\end{tabular} & LGPL   & LGPL     & GPL  & Boost.License \\ \hline
\end{tabular}%
}
\caption{Сравнение фреймворков}
\label{table:frameworkComparison}
\end{table}

\section{Разработка требований}

Выявить сценарии использования проектируемого модуля у будущих пользователей системы.
Оценить объём работы.
Оценить риски, связанные с выбором технологии.

В результате разработки и согласований с заказчиками и пользователями были выделены следующие требования:
\begin{itemize}
    \item Предоставить версию продукта, обладающую минимальной функциональностью
    \item Минимизировать затраты на сопровождение решения
    \item Обеспечить одинаковое поведение при использовании 2D модели или реального робота
\end{itemize}



\section{Архитектура}
\subsection{Робот TRIK}
Поскольку одним из требований к техническому решению было разработка прототипа для начала TODO , то было решено разрабатывать систему в два этапа:
\begin{enumerate}
    \item Разработка минимальной функциональности технического решения
    \begin{itemize}
        \item Обернуть trikControl для использования с PyQt
        \item Реализовать генератор Python в TRIK Studio
    \end{itemize}
    \item Разработка функциональности, аналогичной QtScript в trikScriptRunner
    \begin{itemize}
        \item Использовать PythonQt для встраивания в процесс trikRuntime
        \item Реализовать интерпретацию Python 2D моделью робота
    \end{itemize}
\end{enumerate} 

Такой подход был выбран потому, что хотя PythonQt или Boost.Python и являются наиболее подходящими фреймворками для решения данной задачи, оба фреймворка обладают недостатками, для устранения которых потребовалось бы значительное количество времени. 

Это позволило добиться более естественной интеграции наподобие той, что предлагает QtScript. Более того, пользователям ТРИК, привыкшим работать с JavaScript, не придётся привыкать к абсолютно новой системе. 

\subsection{TRIK Studio}

\section{Особенности реализации}

\section{Документация}
Поскольку образовательные решения TRIK ориентированы в том числе и на школьников средних и старших классов, одним из важным моментов при создании описанной системы было написание базовой документации и создание примеров программ для наглядности. Это упростило работу методистов и позволило первым пользователям быстрее разобраться с системой. TODO

\section{Апробация}


% У заключения нет номера главы
\section*{Заключение}

В рамках данной работы были получены следующие результаты.
\begin{enumerate}
\item Проведён анализ архитектуры существующего ПО образовательных решений ТРИК.
\item Определены требования к системе.
\item Разработана архитектура системы.
\item Система реализована и внедрена в существующие решения ТРИК.
\item Создан прототип пользовательской документации.
\end{enumerate}


В ходе работы промежуточные результаты представлялись докладом на VII Всероссийской конференции "Современное технологическое обучение: От компьютера к роботу", а также докладом на всероссийской конференции "СПИСОК-2017".

Существует несколько направлений дальнейшего развития полученных результатов TODO

\setmonofont[Mapping=tex-text]{CMU Typewriter Text}
\bibliographystyle{ugost2008ls}
\bibliography{diploma.bib}
\end{document}