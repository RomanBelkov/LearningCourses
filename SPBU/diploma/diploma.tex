% Тут используется класс, установленный на сервере Papeeria. На случай, если
% текст понадобится редактировать где-то в другом месте, рядом лежит файл matmex-diploma-custom.cls
% который в момент своего создания был идентичен классу, установленному на сервере.
% Для того, чтобы им воспользоваться, замените matmex-diploma на matmex-diploma-custom
% Если вы работаете исключительно в Papeeria то мы настоятельно рекомендуем пользоваться
% классом matmex-diploma, поскольку он будет автоматически обновляться по мере внесения корректив
%

% По умолчанию используется шрифт 14 размера. Если нужен 12-й шрифт, уберите опцию [14pt]
% \documentclass[14pt]{matmex-diploma}
\documentclass[14pt]{matmex-diploma-custom}

\begin{document}
% Год, город, название университета и факультета предопределены,
% но можно и поменять.
% Если англоязычная титульная страница не нужна, то ее можно просто удалить.
\filltitle{ru}{
    chair              = {Программная инженерия},
    title              = {Интеграция программирования на языке Python в образовательные решения TRIK},
    % Здесь указывается тип работы. Возможные значения:
    %   coursework - Курсовая работа
    %   diploma - Диплом специалиста
    %   master - Диплом магистра
    %   bachelor - Диплом бакалавра
    type               = {bachelor},
    position           = {студента},
    group              = 471,
    author             = {Белков Роман Владимирович},
    supervisorPosition = {ст.\, преп.\,},
    supervisor         = {Я.\,А. Кириленко},
    reviewerPosition   = {OOO "ИнтеллиДжей Лабс"\\ программист},
    reviewer           = {Д.\,А. Мордвинов},
    chairHeadPosition  = {д.\,ф.-м.\,н., профессор},
    chairHead          = {Терехов А.\,Н.},
%   university         = {Санкт-Петербургский Государственный Университет},
%   faculty            = {Математико-механический факультет},
%   city               = {Санкт-Петербург},
%   year               = {2013}
}
\filltitle{en}{
    chair              = {Software engineering},
    title              = {Integration of Python programming with TRIK educational solutions},
    author             = {Roman Belkov},
    supervisorPosition = {senior lecturer},
    supervisor         = {Iakov Kirilenko},
    reviewerPosition   = {IntelliJ Labs Co. Ltd\\ Software Engineer},
    reviewer           = {Dmitry Mordvinov},
    chairHeadPosition  = {professor},
    chairHead          = {Andrey Terekhov},
}
\maketitle
\tableofcontents
% У введения нет номера главы
\section*{Введение}
По своим вычислительным ресурсам робототехнические контроллеры, доступные широкому кругу пользователей, приближаются к показателям персональных компьютеров пятилетней давности. Такая тенденция позволяет постепенно применять в разработке современные методологии и технологии наравне с классическими для контроллеров низкоуровневыми языками и технологиями. Поскольку робототехника активно используется для STEM \cite{stemEducation,stemRobotics} образования, внедрение популярных технологий, использующихся в промышленном программировании, позволит методистам разрабатывать программы обучения, направленные на более широкий круг пользователей. Одной из популярной технологий, набравшей гигантскую популярность в образовательной сфере, является язык Python.

% \subsection*{Python}
Python на данный момент является четвёртым по популярности языком согласно индексу языков программирования TIOBE \cite{indextiobe} на май 2017 года и вторым по популярности языком после Java согласно списку PyPL \cite{indexpypl} на май 2017 года. В последние несколько лет Python стал активно внедряться в образовательные программы. Например, в Массачусетском технологическом институте, одном из ведущих \cite{QSUniRating2016, QSUniRating2017} университетов в области инженерного и технического образования, студентам первого года обучения читают курс "Introduction to Electrical Engineering and Computer Science I" \cite{stemMITCourse}, который представляет собой программирование роботов на языке Python. Другие зарубежные университеты тоже достаточно быстро перешли на использование Python в вводных курсах по программированию \cite{pythonUni}, тем самым обеспечив Python первенство среди языков программирования, использующихся в университетах США.

За трендом, установленным университетами, практически сразу же последовали школы \cite{stemSecCourse, stemSchool}, переходя на Python и внедряя новые курсы обучения программированию на языке Python. В России Python -- один из 5 предложенных в ЕГЭ языков программирования на протяжении уже нескольких лет.

% \subsection*{TRIK}
TRIK\footnote{www.trikset.com} -- это многоцелевой кибернетический контроллер и одноимённый металлический конструктив для прототипирования роботов. Одним из многим применений контроллера является обучение программированию студентов и школьников. Данный контроллер примечателен тем, что обладает достаточными вычислительными мощностями для решения сложных робототехнических задач и реализации ресуркоёмких алгоритмов, а также отсутствием необходимости навыков пайки и электротехники. Для контроллера ТРИК существует среда TRIK Studio \cite{qrealRobots, TRIKStudioTech}, позволяющая облегчить знакомство с робототехникой школьникам младших и средних классов с использованием визуального программирования. Одними из наиболее значимых достоинств среды являются генераторы кода на текстовых языках программирования и интерпретатор текстового кода для 2D модели робота. 

Среда TRIK Studio позволяет преподавателям произвести более плавный переход от визуального программирования к текстовому и впоследствии обучать сложным синтаксическим конструкциям текстовых языков программирования, используя наглядность уже созданных визуальных диаграмм. ПО контроллера TRIK и TRIK Studio образуют программное обеспечение образовательных решений ТРИК, которое на данный момент поддерживает следующие языки программирования: языки платформ Java и .NET, JavaScript, C++, Pascal. Добавив к перечисленным языкам Python, TRIK может считаться идеальной робототехнической платформой для обучения школьников и студентов.


\section*{Постановка задачи}

Целью данной квалификационной работы является разработка и внедрение программного решения, позволяющего использовать язык Python для обучения программированию роботов на базе контроллера ТРИК. Для достижения поставленной цели необходимо выполнить следующие задачи.

\begin{enumerate}
\item Сделать обзор архитетуры существующих образовательных решений ТРИК.
\item Выявить требования к программному решению.
\item Разработать архитектуру программного решения.
\item Реализовать программное решение.
\item Внедрить разработанное решение в образовательные продукты ТРИК.
\end{enumerate}

\section{Обзор технологий}

Как правило, образовательное робототехническое решение состоит из следующих компонентов. TODO буллеты в текст
\begin{itemize}
    \item Методические материалы.
    \item Техническое решение.
        \begin{itemize}
            \item Аппаратная платформа.
            \item Конструктор.
            \item Программное обеспечение.
        \end{itemize}
\end{itemize}

\subsection{ТРИК}
ТРИК — это кибернетический контроллер под управлением операционной системы на основе ядра Linux. Он предназначен для управления роботами, беспилотными летательными аппаратами, средствами передвижения и встраиваемыми устройствами. Несмотря на то, что контроллер ТРИК в первую очередь предназначен для обучения, он может быть использован для решения широкого круга задач — например, построения «умного дома». Это возможно благодаря мощной аппаратной части контроллера: процессорам ARM\footnote{ru.wikipedia.org/wiki/ARM\_(архитектура)}, DSP\footnote{ru.wikipedia.org/wiki/Цифровой\_сигнальный\_процессор}, MSP\footnote{ru.wikipedia.org/wiki/MSP430}.

В образовательных решениях ТРИК программное обеспечение представлено следующими компонентами:
\begin{itemize}
    \item Библиотека времени исполнения trikRuntime на роботе
    \item Среда программирования роботов TRIK Studio
    \item Интерпретатор 2D модели для проверки задач на Stepik
\end{itemize}

\subsubsection{trikRuntime}
trikRuntime -- это набор библиотек, реализующий часть технической составляющей ТРИК. Хотя для ТРИК и созданы другие библиотеки времени исполнения, например, trik-sharp \cite{KirsanovSECR, KirsanovDiploma} или TrikKotlin \cite{BelkovYearlyProject}, trikRuntime остаётся наиболее популярной и широко используемым фреймворком. Полная архитектура trikRuntime представлена на рисунке \ref{trikRuntime} (страница \pageref{trikRuntime}).

\begin{figure}[h]
	\includegraphics[width=\textwidth]{images/trikRuntime.jpg}
	\caption{Архитектура trikRuntime}
	\label{trikRuntime}
\end{figure}

Библиотека trikControl предоставляет интерфейс для программирования роботов с помощью языков JavaScript или C++ с использованием Qt. Ниже перечислены некоторые классы trikControl: Brick — класс, отвечающий за контроллер, инициализирующий периферию робота и дающий к ней доступ, Sensor, ServoMotor и PowerMotor — классы, отвечающие за работу с сенсорами, сервомоторами и силовыми моторами, соответственно. Каждый из них даёт возможность прочитать или выставить значение, что выполняется путём записи или чтения значения в файле, отвечающем за соответствующее устройство.

% Идеал -- QtScript\footnote{doc.qt.io/qt-4.8/qtscript-module.html}, переименованный в QJSEngine\footnote{doc.qt.io/qt-5/qjsengine.html} с версии Qt 5.0 и уже используемый в основном фреймворке для контроллера ТРИК.

Подробно рассмотрим, как исполняется код на языке JavaScript с помощью компонентов trikRun и trikScriptRunner. Архитектуру компонентов находится на рисунке на странице TODO. trikScriptRunner использует технологию QtScript\footnote{doc.qt.io/qt-4.8/qtscript-module.html} (с версии Qt 5.0 -- QJSEngine\footnote{doc.qt.io/qt-5/qjsengine.html}). QtScript -- это язык программирования, который реализует стандарт ECMAScript\footnote{https://ru.wikipedia.org/wiki/ECMAScript} с расширением в виде поддержки Qt. QtScript практически не отличается от JavaScript с точки зрения пользователя и в то же время позволяет без труда переиспользовать существующий C++ код, написанный с использованием Qt. 

\subsubsection{TRIK Studio}
TRIK Studio -- это среда визуального и текстового программирования, поддерживающая наиболее популярные в России и Европе образовательные робототехнические платформы: Lego NXT, Lego EV3, TRIK, некоторые версии STM32. В активной стадии разработки -- поддержка большего количества платформ. Поддержка аппаратных платформ осуществляется специальными плагинами. 

Плагин являет собой генератор или интерпретатор текстового языка программирования. Модульная архитектура плагинов TRIK Studio позволяет легко переиспользовать готовые компоненты, расширяя функциональность при необходимости. Среда предоставляет большую часть необходимого для генераторов: преобразование графа потока управления к представлению, пригодному для генерации структурного кода, подсветку синтаксиса на базе библиотеки QScintilla\footnote{www.riverbankcomputing.com/software/qscintilla/intro}. Фактически, только генерацию текстового представления программы и поддержку компилятора (для компилируемых языков) для генераторов необходимо реализовывать вручную. Разнообразие уже созданных генераторов и интерпретаторов языков позволяют без труда понять, как именно создать новый плагин для целевого языка.

Помимо плагинов, поддерживающих генерацию или интерпретацию языков программирования, TRIK Studio обладает инструментами автоматизации и отладки: например, пользователь может из среды загрузить и запустить программу на роботе, а после запуска программы на роботе получать вывод программы в консоль TRIK Studio.

\subsubsection{TRIK Stepik checker}
Для популярной платформы онлайн-образования Stepik\footnote{stepik.org} существует вводный курс в робототехнику "Первый шаг в робототехнику" \footnote{stepik.org/462} от методистов ТРИК. Для этого курса потребовалось создать специализированную систему, способную проверять решения пользователей на корректность. Так появился TRIK Stepik checker, который является изолированной частью

TODO это вообще надо? Как верно оформить мысль?

\subsection{Фреймворки взаимодействия Python и C++}

Существует большое количество фреймворков, с использованием которых можно решить задачу использования существующего кода на C++ из языка Python. Поскольку ПО контроллера ТРИК и среда TRIK Studio являются свобободно распространяемым программным обеспечением, то в данной работе рассматривались только следующие фреймворки: PyQt, PySide, PythonQt, Boost.Python, SWIG.

Прежде, чем приступить к рассмотрению фреймворков взаимодействия, были выделены следующие требования, которым должен удовлетворять выбранный фреймворк.

\begin{itemize}
    \item Возможность поддержать исполнение Python из C++ приложения
    \item Поддержка Qt
    \item Активная поддержка фреймворка компанией или сообществом
    \item Доступная документация
\end{itemize}

\subsubsection{PyQt}

PyQt \cite{pyqt} -- это набор обёрток для фреймворка Qt, который позволяет использовать большую часть функциональности Qt в Python, тем самым достигая 

PyQt очень популярен и хорош для переиспользования существующих библиотек, написанных на C++ с использованием Qt из Python. PyQt не предоставляет возможностей для встраивания частей на языке Python в Qt-приложение. Для удобного переиспользования уже существующего кода на C++ существует инструмент SIP\footnote{www.riverbankcomputing.com/software/sip/intro}, который по специальным sip-файлам генерирует код обёрток на языке C++. Однако sip-файлы весьма неудобны тем, что они являются почти точными копиями заголовочных файлов. Это приводит к тому, что разработчикам приходиться изменять sip-файлы при каждом изменении C++ кода библиотеки. Более того, SIP обладает весьма слабым парсером языка C++, что не позволяет поддерживать некоторые конструкции динамически. Таким образом, приходиться прибегать к инъекциям дописанного руками кода в уже сгенерированный код, что является достаточно плохой практикой в силу затратности такого подхода и высокой вероятности допустить ошибку.

\subsubsection{PySide}
PySide -- это набор обёрток для фреймворка Qt. Проект PySide стартовал как аналог PyQt и практически не отличается от PyQt. Единственное значимое отличие -- PyQt распространяется под коммерческой или GPLv3 лицензией, в то время как PySide распространяется под LGPL лицензией, что позволяет использовать его с большей свободой. 

Плюсом фреймворка является то, что Qt Company\footnote{en.wikipedia.org/wiki/The\_Qt\_Company} официально поддерживает PySide, что обеспечивает активную разработку фреймворка. 

Несмотря на это, документация у проекта всё равно находится в весьма начальной стадии. Также стоит отметить, что PySide поддерживает на данный момент меньшую часть фреймворка Qt, чем PyQt.

\subsubsection{PythonQt}
TODO
Плюсы:
Возможна интерпретация Python из C++ приложения
Поддержка Qt
Проблемы:
Часть обёрток придётся писать руками

PythonQt -- это библиотека для фреймворка Qt, нацеленный на встраивание Python в С++/Qt приложение. PythonQt разрабатывался

PythonQt is a dynamic Python binding for the Qt framework. It offers an easy way to embed the Python scripting language into your C++ Qt applications.

The focus of PythonQt is on embedding Python into an existing C++ application, not on writing the whole application completely in Python.

\subsubsection{Boost.Python}
Boost.Python \cite{abrahams2003boost} -- это библиотека для взаимодействия C++ и Python, позволяющая использовать скрипты на языке Python как часть программы, написанной на C++. 

Основная цель Boost.Python -- позволить пользователям переиспользовать C++ в Python, используя лишь компилятор C++. Это отличает его от, например, SIP, который использует специальный язык для создания промежуточной коммуникации. При этом при создании Boost.Python учитывались основные различия между двумя языками и были представлены интерфейсы для удобного взаимодействия.

Boost.Python предлагает возможность интерпретации Python

Для автоматизации процесса создания обёрток к коду существуют следующие инструменты: Pyste и Py++. Pyste\footnote{www.boost.org/doc/libs/1\_60\_0/libs/python/pyste/doc/running\_pyste.html} -- генератор обёрток для Boost.Python, разрабатывавшийся как часть фреймворка Boost.Python. Разработка была прекращена в 2011 году. Pу++\footnote{bitbucket.org/ompl/pyplusplus} -- генератор обёрток для Boost.Python, используемый в популярной библиотеке планирования движения OMPL\cite{sucan2012theOMPL}. В отличие от Pyste, Py++ поддерживается сообществом и, следовательно, может быть использован при разработке новых проектов с использованием Boost.Python.
% Более того, Pyste использует GCC-XML\cite{gccxml}, разработка которого официально остановилась в 2015 году\footnote{gccxml.github.io/HTML/News.html}. 


Однако, поскольку Boost.Python разрабатывался для работы с чистым С++ кодом, то в нём отсутствует поддержка фреймворка Qt. Это не позволяет использовать сигналы и слоты Qt из Python, что является недостатком. 

% The primary goal of Boost.Python is to allow users to expose C++ classes and functions to Python using nothing more than a C++ compiler. In broad strokes, the user experience should be one of directly manipulating C++ objects from Python.

% However, it's also important not to translate all interfaces too literally: the idioms of each language must be respected. For example, though C++ and Python both have an iterator concept, they are expressed very differently. Boost.Python has to be able to bridge the interface gap.

% It must be possible to insulate Python users from crashes resulting from trivial misuses of C++ interfaces, such as accessing already-deleted objects. By the same token the library should insulate C++ users from low-level Python 'C' API, replacing error-prone 'C' interfaces like manual reference-count management and raw PyObject pointers with more-robust alternatives.

% Support for component-based development is crucial, so that C++ types exposed in one extension module can be passed to functions exposed in another without loss of crucial information like C++ inheritance relationships.

% Finally, all wrapping must be non-intrusive, without modifying or even seeing the original C++ source code. Existing C++ libraries have to be wrappable by third parties who only have access to header files and binaries.

\subsubsection{SWIG}
TODO выкинуть SWIG

\subsection{Выбор фреймворка}

Таблица \ref{table:frameworkComparison} на странице \pageref{table:frameworkComparison}

Сначала PyQt, потом PythonQt

% \begin{table}[]
% \centering
% \resizebox{\textwidth}{!}{%
% \begin{tabular}{|l|c|c|c|c|c|}
% \hline
%                                                                               & PyQt & PySide & PythonQt & SWIG & Boost.Python \\ \hline
% поддержка Qt                                                                  & +    & +      & +        & +/-  & -            \\ \hline
% \begin{tabular}[c]{@{}l@{}}поддержка фреймворка\\ разработчиками\end{tabular} & +    & -      & -        & +    & +            \\ \hline
% доступная документация                                                        & +    & -      & -        & +    & +            \\ \hline
% \end{tabular}%
% }
% \caption{Сравнение фреймворков}
% \label{table:frameworkComparison}
% \end{table}

\begin{table}[]
\centering
\resizebox{\textwidth}{!}{%
\begin{tabular}{|l|c|c|c|c|c|}
\hline
                                                                          & PyQt                                                          & PySide & PythonQt & SWIG & Boost.Python  \\ \hline
встраивание Python в C++                                                  & -                                                             & -      & +        & -    & +             \\ \hline
поддержка Qt                                                              & +                                                             & +      & +        & -    & -             \\ \hline
\begin{tabular}[c]{@{}l@{}}поддержка фреймворка\end{tabular} & +                                                             & -      & -        & +    & +             \\ \hline
хорошая документация                                                      & +                                                             & -      & -        & +    & +             \\ \hline
обёртка С++ для Python                                                    & +                                                             & +      & -        & +    & +             \\ \hline
лицензия                                                                  & \begin{tabular}[c]{@{}c@{}}GPL + \\ коммерческая\end{tabular} & LGPL   & LGPL     & GPL  & Boost.License \\ \hline
\end{tabular}%
}
\caption{Сравнение фреймворков}
\label{table:frameworkComparison}
\end{table}

\section{Разработка требований}

TODO
Выявить сценарии использования проектируемого модуля у будущих пользователей системы.
Оценить объём работы.
Оценить риски, связанные с выбором технологии.

В результате разработки и согласований с заказчиками и пользователями были выделены следующие требования:
\begin{itemize}
    \item Предоставить версию продукта, обладающую минимальной функциональностью
    \item Минимизировать затраты на сопровождение решения
    \item Обеспечить одинаковое поведение при использовании 2D модели или реального робота
\end{itemize}



\section{Архитектура}

Поскольку одним из требований к техническому решению было разработка прототипа для начала TODO , то было решено разрабатывать систему в два этапа:
\begin{enumerate}
    \item Разработка минимальной функциональности технического решения
    \begin{itemize}
        \item Обернуть trikControl для использования с PyQt
        \item Реализовать генератор Python в TRIK Studio
    \end{itemize}
    \item Разработка функциональности, аналогичной QtScript в trikScriptRunner
    \begin{itemize}
        \item Использовать PythonQt для встраивания в trikRuntime
        \item Реализовать интерпретацию Python 2D моделью робота
    \end{itemize}
\end{enumerate} 

Такой подход был выбран потому, что хотя PythonQt или Boost.Python и являются наиболее подходящими фреймворками для решения данной задачи, оба фреймворка обладают недостатками, для устранения которых потребовалось бы значительное количество времени. 

Это позволило добиться более естественной интеграции наподобие той, что предлагает QtScript. Более того, пользователям ТРИК, привыкшим работать с JavaScript, не придётся привыкать к системе с новым usage flow TODO. 

\subsection{Робот TRIK}

\subsection{TRIK Studio}

\subsection{TRIK Stepik Checker}


% \section{Особенности реализации}

\section{Документация}
Поскольку образовательные решения TRIK ориентированы в том числе и на школьников средних и старших классов, одним из важным моментов при создании описанной системы было написание базовой документации и создание примеров программ для наглядности. Это упростило работу методистов и позволило первым пользователям быстрее разобраться с представленной системой. TODO

\section{Апробация}


% У заключения нет номера главы
\section*{Заключение}

В рамках данной работы были получены следующие результаты.
\begin{enumerate}
\item Сделан обзор архитектуры существующего ПО образовательных решений ТРИК.
\item Определены требования к программному решению.
\item Разработана архитектура программного решения.
\item Выполнена реализация программного решения.
\item Разработанное программное решение внедрено в образовательные решения ТРИК.
\end{enumerate}



В ходе работы промежуточные результаты представлялись докладом на VII Всероссийской конференции "Современное технологическое обучение: От компьютера к роботу", а также докладом на всероссийской конференции "СПИСОК-2017".

Существует несколько направлений дальнейшего развития полученных результатов. Возможно создание аналога QJSEngine для языка Python TODO

\setmonofont[Mapping=tex-text]{CMU Typewriter Text}
\bibliographystyle{ugost2008ls}
\bibliography{diploma.bib}
\end{document}